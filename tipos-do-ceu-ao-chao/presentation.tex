\documentclass{beamer}

\usepackage[brazil]{babel}
\usepackage[T1]{fontenc}
\usepackage[utf8]{inputenc}

\usepackage{listings}

\usetheme{Hannover}
\usecolortheme{crane}

\title{Tipos do Céu ao Chão}
\author{Hashimoto}
\date{?? de ??? de 2022}

\begin{document}

\begin{frame}
    \titlepage
\end{frame}

\begin{frame}
    \frametitle{ToC}
    \tableofcontents[hideallsubsections]
\end{frame}

\section{Tipos}
\begin{frame}
    \frametitle{Intro}
\end{frame}

\subsection{Notação}
\begin{frame}
    \frametitle{Notação}
\end{frame}

\section{Tipos - Caso Recursivo}
\subsection{Produto}
\begin{frame}
    \frametitle{Produto \(A \times B\)}
    \begin{itemize}
        \item Construtor
            \[
                A \to B \to A \times B
            \]
        \item Destrutor
            \[
                A \times B \to (A \to B \to C) \to C
            \]
    \end{itemize}
\end{frame}

\begin{frame}[fragile]
    \frametitle{Exemplos: \(A \times B\)}
    \begin{itemize}
        \item C:
        \begin{lstlisting}[language=C]
struct produto {
    int A;
    int B;
};
        \end{lstlisting}
        \item Java:
        \begin{lstlisting}[language=Haskell]
class Produto {
    private int A;
    private int B;
}
        \end{lstlisting}
        \item Haskell:
        \begin{lstlisting}[language=Haskell]
data Produto = P { a :: Int, b :: Int }
        \end{lstlisting}
    \end{itemize}
\end{frame}

\subsection{Soma}
\begin{frame}
    \frametitle{Soma \(A + B\)}
    \begin{itemize}
        \item Construtores
            \[
                left : A \to A + B
            \]
            \[
                right : B \to A + B
            \]
        \item Destrutor
            \[
                A + B \to (A \to C) \to (B \to C) \to C
            \]
    \end{itemize}
\end{frame}

\begin{frame}
    \frametitle{Exemplos: \(A + B\)}
\end{frame}

\subsection{Exponencial}
\begin{frame}
    \frametitle{Exponencial \(B^A\)}
    \begin{itemize}
        \item Construtor
            \[
                (A \to B) \to B^A
            \]
        \item Destrutor
            \[
                eval : B^A \to A \to B
            \]
    \end{itemize}
\end{frame}

\begin{frame}
    \frametitle{Exemplos: \(B^A\)}
\end{frame}

\section{Tipos - Casos Base}
\begin{frame}
    \frametitle{Placeholder}
\end{frame}

\subsection{Void}
\begin{frame}
    \frametitle{Void \(Void\)}
\end{frame}

\subsection{Unity}
\begin{frame}
    \frametitle{Unity \(()\)}
\end{frame}

\subsection{Bool}
\begin{frame}
    \frametitle{Bool}
\end{frame}

\subsection{Int}
\begin{frame}
    \frametitle{Int (I32)}
\end{frame}

\section{Tipos Paramétricos}
\begin{frame}
    \frametitle{Tipos Paramétricos}
\end{frame}

\subsection{Maybe}
\begin{frame}
    \frametitle{Maybe}
\end{frame}

\subsection{Either}
\begin{frame}
    \frametitle{Either}
\end{frame}

\section{Classes de Tipos}
\begin{frame}
    \frametitle{Classes de Tipos}
\end{frame}

\subsection{Funtores}
\begin{frame}
    \frametitle{Funtores}
\end{frame}

\subsection{Implementação}
\begin{frame}
    \frametitle{Funtores}
\end{frame}

\section{Outros Tipos}
\begin{frame}
    \frametitle{Outros Tipos (Fora do escopo)}
\end{frame}

\subsection{Tipos Dependentes}
\begin{frame}
    \frametitle{Tipos Dependentes}
\end{frame}

\subsection{Tipos Lineares}
\begin{frame}
    \frametitle{Tipos Lineares}
\end{frame}

\begin{frame}
    \frametitle{Placeholder}
\end{frame}

\begin{frame}
    \frametitle{FIM}
\end{frame}

\end{document}
